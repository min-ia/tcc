\chapter[Introdução]{Introdução}
% \addcontentsline{toc}{chapter}{Introdução}

Existe no mundo um novo cenário emergente no qual, empresas nascentes,
denominadas \textit{startups}, produzem inovação ao ponto de serem capazes
de competir com empresas já estabelecidas no mercado \cite{CapacidadeDeInovacao}.
O ambiente de negócios está mudando devido a fatores como globalização, política
e tecnologia. Neste novo contexto, tudo que se conhece dos negócios são
hipóteses e um constante alto grau de risco \cite{Akiyoshi}.

Nesse cenário, surgiu a \textit{startup} Rua Dois Tecnologias. Com uma série
de hipóteses sobre o mercado imobiliário, a empresa visa inovar esse ramo que
é ainda muito burocrático e carente de melhorias \cite{LouisaXu}. E nessa
euforia por inovação, a \textit{startup} tem em seu modelo de negócios a
constante busca pela validação rápida de ideias mas de uma forma que ela
esteja preparada pra crescer e se expandir rapidamente, conforme os princípios
de repetibilidade e escalabilidade\footnote{Os princípios de Steve Blank sobre
\textit{startups} serão abordados com maior profundidade na 
\autoref{sec:OQueEUmaStartup}.} de \citeonline{SteveBlankFirstPrinciples} para
\textit{startups}.

Nessa busca por escalabilidade a empresa tomou uma série de decisões tecnológicas
baseadas no que seria escalável, mas sem uma análise coerente com o sistema
proposto e com o contexto de validação de ideias que a empresa vivenciava.
A exemplo está a escolha de utilizar uma arquitetura de microsserviços sem conhecer
de fato a demanda dos serviços que seriam prestados, ocasionando dificuldades para
a empresa de modificar e evoluir esse sistema com a rapidez que é desejada para uma
\textit{startup}.

Assim surgem as indagações do presente trabalho. Neste contexto das \textit{startups}
o software produzido é incessantemente alterado em busca da melhor solução para um
determinado problema, é possível preparar esse software de forma que ele possa ser
escalável? Se for possível, quais práticas e recomendações podem ser aplicadas para
alcançar este objetivo?

Com foco nessas questões, este trabalho almeja realizar um estudo de caso sobre o
desenvolvimento de software na \textit{startup} Rua Dois, visando avaliar as questões
apresentadas e direcionar esta empresa para um caminho mais consolidade em relação
aos seus objetivos de crescimento do sistema.

\section{Justificativa}

No livro \textit{Software Architeture in Pratice}, os autores \citeonline{Bass2015:SoftwareArchitetureInPratice}
apresentam a arquitetura de software como a ponte capaz de auxiliar as empresas a atingirem seus
objetivos de negócio. Objetivos estes que por diversas vezes tendem a ser bastante abstratos,
cabendo a arquitetura de software o grande papel de concretizar esses objetivos em sistemas
\cite{Bass2015:SoftwareArchitetureInPratice}.

Na intenção de construir essa ponte, frequentemente times de software se encontram ansiosos para
adotar uma arquitetura de microsserviços levando aos seus sistemas as grandes vantagens que essa
arquitetura é capaz de trazer, contudo o grande prêmio oferecido vêm acompanhado de vários custos e
riscos que os desenvolvedores tendem a não considerar quando optam pela utilização de
microsserviços \cite{MartinFowler:MicroservicePremium}.

Do outro lado, a arquitetura monolítica, comumente adotada por diversas aplicações, é inicialmente
simples e agradável de manter, mas tende a crescer a sua complexidade mediante a sua inerente
característica de ser acoplada \cite{StefanTilkov:DontStartWithAMonolith}.

Diante deste contexto, o presente trabalho justifica-se por buscar esclarecer as diferenças entre
ambos os estilos arquiteturais, monolítico e microsserviços, visando fornecer embasamento para equipes
de desenvolvimento de software analisarem de forma mais consciente o contexto em que os projetos
estão inseridos e como a escolha de determinado tipo arquitetural pode influenciar, positivamente ou
negativamente, o alcance dos objetivos mencionados.

\section{Objetivos}

\subsection{Objetivo Geral}
\label{sec:ObjetivoGeral}

Este trabalho tem como objetivo geral realizar uma análise comparativa acerca dos estilos
arquiteturais de software monolítico e microsserviços com o intuito de compreender as
características de cada estilo e em quais contextos cada um é melhor aplicado. Para tal, pretende-se
realizar uma pesquisa exploratória baseada em revisão bibliográfica e em relatos de casos reais nos
quais empresas optaram por migrar de uma arquitetura monolítica para uma arquitetura de
microsserviços ou vice-versa.

\subsection{Objetivos Específicos}

As seguintes metas foram levantadas visando atingir o objetivo geral do presente
trabalho:

  \begin{enumerate}
      \item Compreender o que é arquitetura de software e como ela influência as características dos
          sistemas de software;
      \item Compreender o que é uma arquitetura monolítica e suas principais características;
      \item Compreender o que é uma arquitetura de microsserviços e suas principais características;
      \item Explorar diferentes visões apresentadas na bibliografia acerca de ambos os estilos
          arquiteturais;
      \item Explorar relatos de casos reais nos quais empresas optaram por migrar de uma arquitetura
          para a outra, com o intuito de compreender as dificuldades e vantagens encontradas por
          elas em cada estilo arquitetural;
      \item Entrevistar uma \textit{startup} brasileira X que optou por realizar a migração de
          microsserviços para um sistema monolítico, compreendendo o porquê dessa decisão e como
          esta impactou a \textit{startup};
      \item Agrupar e classificar as informações obtidas por meio da revisão bibliográfica, dos
          relatos reais e da entrevista realizada, com o objetivo de analisar e comparar cada estilo
          arquitetural;
      \item Elencar quais são as características de cada estilo arquitetural e com base nelas
          compreender em quais contextos cada estilo arquitetural é melhor aplicado.
  \end{enumerate}
