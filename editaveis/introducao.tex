\chapter[Introdução]{Introdução}

Há cerca de dez anos atrás, a Amazon e a Netflix se destacavam no mercado de software trazendo ao
mundo uma tendência arquitetural: os microsserviços. A grande promessa era que essa
arquitetura era capaz de prover escalabilidade, resiliência e diversos outros benefícios para os
problemas comumente enfrentados pelas empresas com seus sistemas legados e monolíticos. Assim,
dava-se início a uma nova corrida entre as empresas: Spotify, Coca Cola e muitos outros
grandes nomes começaram a migrar a sua arquitetura para esse novo padrão tão promissor
\cite{Aleksandra2019}.

Dez anos após o início dessa grande tendência no mundo de software, têm se observado o movimento
inverso: um crescente interesse em cima da arquitetura monolítica, em particular, dos monolíticos
modulares \cite{QTrends}. A realidade é que muitas empresas têm se aventurado na arquitetura de
microsserviços e se deparado com uma séries de desafios além do que estavam preparadas para
enfrentar. Nas palavras de \citeonline{MartinFowler:MicroserviceTradeOffs}:

\begin{quotation}{MartinFowler:MicroserviceTradeOffs}{transleted=true}
    Muitas equipes de desenvolvimento descobriram no estilo arquitetural dos microsserviços uma
    abordagem superior a uma arquitetura monolítica. Mas outras equipes descobriram que eles
    são um fardo que enfraquece a produtividade. \footnote{Texto original: \textit{Many
    development teams have found the microservices architectural style to be a superior approach
    to a monolithic architecture. But other teams have found them to be a productivity-sapping
    burden.}}
\end{quotation}

Assim, nos deparamos com uma realidade na qual existe uma frustração antiga das empresas em trabalhar
com a arquitetura monolítica devido as dificuldades de manutenibilidade que eventualmente essa arquitetura
acaba demonstrando, e que na tentativa de sair desses problemas e conquistar a modularidade e todos os outros
benefícios oferecidos pelos microsserviços essas empresas acabam se frustrando novamente mediante os desafios
desse estilo arquitetural.

Diante desta situação, o presente trabalho visa realizar uma análise comparativa entre os dois estilos
arquiteturais, monolítico e microsserviços, com o intuito de compreender as diferenças entre cada estilo
e quais são os fatores que influenciam, positivamente ou negativamente, cada uma dessas abordagens
arquiteturais.

\section{Justificativa}
\label{justificativa}

No livro \textit{Software Architeture in Pratice}, os autores \citeonline{Bass2015:SoftwareArchitetureInPratice}
apresentam a arquitetura de software como a ponte capaz de auxiliar as empresas a atingirem seus
objetivos de negócio. Objetivos estes que por diversas vezes tendem a ser bastante abstratos,
cabendo a arquitetura de software o grande papel de concretizar esses objetivos em sistemas
\cite{Bass2015:SoftwareArchitetureInPratice}.

Na intenção de construir essa ponte, frequentemente times de software se encontram ansiosos para
adotar uma arquitetura de microsserviços levando aos seus sistemas as grandes vantagens que essa
arquitetura é capaz de trazer, contudo o grande prêmio oferecido vêm acompanhado de vários custos e
riscos que os desenvolvedores tendem a não considerar quando optam pela utilização de
microsserviços \cite{MartinFowler:MicroservicePremium}.

Do outro lado, a arquitetura monolítica, comumente adotada por diversas aplicações, é inicialmente
simples e agradável de manter, mas tende a crescer a sua complexidade mediante a sua inerente
característica de ser acoplada \cite{StefanTilkov:DontStartWithAMonolith}.

Diante deste contexto, o presente trabalho justifica-se por buscar esclarecer as diferenças entre
ambos os estilos arquiteturais, monolítico e microsserviços, visando fornecer embasamento para equipes
de desenvolvimento de software analisarem de forma mais consciente o contexto em que os projetos
estão inseridos e como a escolha de determinado tipo arquitetural pode influenciar, positivamente ou
negativamente, o alcance dos objetivos de negócio.

\section{Objetivos}

\subsection{Objetivo Geral}
\label{sec:ObjetivoGeral}

Este trabalho tem como objetivo geral realizar uma análise comparativa acerca dos estilos
arquiteturais de software monolítico e microsserviços com o intuito de compreender as
características de cada estilo e em quais contextos cada um é melhor aplicado. Para tal, fez-se
uma pesquisa exploratória baseada em revisão bibliográfica e em relatos de casos reais nos
quais empresas optaram por migrar de uma arquitetura monolítica para uma arquitetura de
microsserviços ou vice-versa.

\subsection{Objetivos Específicos}

As seguintes metas foram levantadas visando atingir o objetivo geral do presente
trabalho:

  \begin{enumerate}
      \item Compreender o que é arquitetura de software e como ela influencia as características dos
          sistemas de software;
      \item Compreender o que é uma arquitetura monolítica e suas principais características;
      \item Compreender o que é uma arquitetura de microsserviços e suas principais características;
      \item Explorar diferentes visões apresentadas na bibliografia acerca de ambos os estilos
          arquiteturais;
      \item Explorar relatos de casos reais nos quais empresas optaram por migrar de uma arquitetura
          para a outra, com o intuito de compreender as dificuldades e vantagens encontradas por
          elas em cada estilo arquitetural;
      \item Entrevistar a \textit{startup} RuaDois que optou por realizar a migração de
          microsserviços para um sistema monolítico, compreendendo o porquê dessa decisão e como
          esta impactou a \textit{startup};
      \item Agrupar e classificar as informações obtidas por meio da revisão bibliográfica, dos
          relatos reais e da entrevista realizada, com o objetivo de analisar e comparar cada estilo
          arquitetural;
      \item Elencar quais são as características de cada estilo arquitetural e com base nelas
          compreender em quais contextos cada estilo arquitetural é melhor aplicado.
  \end{enumerate}
