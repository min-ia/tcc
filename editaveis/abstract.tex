\begin{resumo}[Abstract]
 \begin{otherlanguage*}{english}
    The monolithic architecture brings within software engineering a historic of
    legacy systems and teams frustrated with the complexity of maintaining these
    systems. On the other hand, the microservice architecture is capable of providing
    a series of benefits desired by several companies. This context leads these companies
    to choose to adopt or migrate their systems to the microservice architecture. However,
    this migration carried out in disagreement with the company's business objectives and
    without the proper planning, leads these systems to the failure of the architectural
    model. With this in mind, the present study carried out exploratory research on the
    monolithic and microservice architectural styles, building a mental map with the
    factors that affect each architectural model. In the end, these factors were analyzed
    in four case studies of companies that chose to migrate from one architectural model
    to another. Finally, it was concluded that the monolithic architecture is suitable for
    discovering the domain but this architecture tends to lose maintainability and
    evolutionability as the code base grows. While the microservice architecture tends to
    be more sustainable, however, it has a high cost and requires domain over the
    problematic and technologies.

   \vspace{\onelineskip}
 
   \noindent 
   \textbf{Keywords}: Microservices. Monolithic Systems. Software Architecture.
 \end{otherlanguage*}
\end{resumo}
