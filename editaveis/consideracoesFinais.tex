\chapter{Considerações Finais}

O presente trabalho dedicou-se a realizar uma comparação entre os estilos arquiteturais monolítico e
microsserviços, por meio de uma pesquisa exploratória com base principalmente nos autores Martin Fowler,
Sam Newman, Mark Richards e Neal Ford, dentre outros. Traçando um paralelo com os casos de estudo do
KN Login, Otto e Segment objetivando abordar juntamente uma visão prática sobre os modelos arquiteturais.

Notou-se que, em geral, os autores estudados apresentam uma concepção parecida a respeito do
uso de cada estilo arquitetural mendiante o contexto da aplicação e que os casos de estudo
corroboram com essa visão exposta pelos autores.

A perspectiva final sobre os estilos arquiteturais analisados é que a arquitetura monolítica tende a
ter um menor custo e por isso é mais recomendada para fase inicial do sistema como um meio de
descobrir a aplicação, porém é uma arquitetura que tende a perder a sua manutenibilidade e
evolucionabilidade a medida que a base de código cresce e por esta razão alguns autores a indicam como uma
arquitetura de sacrifício, de forma que após explorar a problemática da aplicação essa arquitetura
dê espaço para um modelo arquitetural mais sustentável.

No caso dos microsserviços, este já tende a ser um modelo arquitetural mais sustentável, porém exige
um custo maior de construção e manutenção além de domínio sobre o problema, o que faz com que nem todas
as empresas estejam preparadas para adotar tal estilo arquitetural ou alcançar todos os benefícios
que este modelo pode proporcionar.

Diante das limitações de tempo do presente estudo, focou-se bastante na análise dos aspectos
arquiteturais abordados pelos referenciais teóricos, entretanto, existem outros pontos levantados
pelos casos práticos, como a curva de aprendizado dentro de cada estilo arquitetural, ou mesmo pontos que
não foram explorados nesta análise e são pouco explorados na bibliografia a respeito do tema, a exemplo,
a influência do tamanho do time na escolha do estilo, questões de alocação de recursos, disponibilidade
da aplicação, aspectos de segurança, processo de \textit{debugger}, documentação, etc., pontos que abrem
espaço para futuras pesquisas e aprofundamento no tema.

Por fim, deseja-se que engeheiros de software ao lerem o presente estudo obtenham maior embasamento
para analisar o contexto no qual estão inseridos no momento de optar por utilizar uma arquitetura de
microsserviços ou monolítica, tomando uma decisão muito mais sólida e alinhada as expectivas e objetivos
de negócio de cada empresa.
