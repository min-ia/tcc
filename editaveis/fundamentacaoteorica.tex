\chapter{Fundamentação Teórica}

Neste capítulo será apresentado as bases teóricas que permeiam o escopo
do presente trabalho. Partindo do entendimento sobre o contexto de
\textit{startups} até as questões relacionadas a escalabilidade de softwares.
As seções estão dispostas em:

  \begin{enumerate}
    \item \textbf{\textit{Startups}:} contexto, organização e propósitos de uma 
      \textit{startup};
    \item \textbf{Desenvolvimento de Software:} abordagem sobre boas práticas
    de desenvolvimento de software e como estas estão sendo abordadas no contexto
    de \textit{startups};
    \item \textbf{Escalabilidade de Software:} definição de Escalabilidade e 
      possíveis formas de aplicação.
  \end{enumerate}

\section{\textit{Startups}}

No livro \textit{A Revolução das Startups} de \citeonline{ARevolucaoDasStartups},
o autor descreve uma Geração Y, nascida entre 1980 e 2000, detentora de um
espírito jovem em busca de ser feliz fazendo algo que seja realmente
impactante para a sociedade. Diferente das gerações passadas, a Geração Y
possui em suas mãos a Internet e o conhecimento para utilizar esta
ferramenta em todo o seu potencial.

Essa geração leva a sociedade a um contexto no qual as pessoas estão sempre
se questionando se existe outra forma de fazer, se podemos fazer melhor e
se somos capazes de resolver algum problema \cite{ARevolucaoDasStartups}.
A partir dessas indagações nascem \textit{startups} revoluniciárias, como
o Waze\footnote{Empresa que conecta motoristas visando melhorar o tráfego
dentro das cidades. Saiba mais em: \url{https://www.waze.com}}, lançado em
2009 com a proposta de mudar a forma das pessoas se locomoverem, deixando 
de lado a maneira tradicional de se usarem mapas e fazendo com que as 
pessoas interajam e contribuam para o próprio mapeamento
\cite{NepomucenoSucessoDoWaze}.

\subsection{O que é uma \textit{Startup}?}
\label{sec:OQueEUmaStartup}

\begin{quotation}{SteveBlankFirstPrinciples}{transleted=true}
Uma \textit{startup} é uma organização formada para buscar um modelo de 
negócios repetível e escalável.\footnote{Texto original: \textit{A startup is an
organization formed to search for a repeatable and scalable business model.}}
\end{quotation}

A definição apresentada por Steve Blank traz características
fundamentais para entender o conceito de uma \textit{startup}. Começando
pela organização que representa um grupo de pessoas alinhadas em prol
de um mesmo objetivo. Esse grupo de pessoas juntamente com seus ideais
são a essência de uma \textit{startup}, afinal, são elas que fazem
o negócio fluir somando as habilidades individuais de cada um
\cite{ARevolucaoDasStartups}.

O modelo de negócio é a descrição de como a empresa cria, entrega e captura
valor, exibindo todos os fluxos entre as diferentes partes da empresa,
como o produto é distribuído para os clientes, como o dinheiro retorna, a
estruturas de custos e a interação com outras empresas parceiras. Uma
\textit{startup} consiste essencialmente em uma organização criada para
procurar um modelo de negócios, iniciando com uma visão sobre o produto e
uma série de hipóteses sobre o modelo de negócios
\cite{SteveBlankFirstPrinciples}.

Segundo \citeonline{ARevolucaoDasStartups}, ser repetível significa que 
a ideia deve ser facilmente replicável em outras regiões, países ou até
mesmo outros setores, visando diversificar os ganhos. Ou seja, quanto mais
pessoas utilizam o serviço, melhor para as \textit{startups}.

Atingir o objetivo de ser repetível carrega junto a missão de ser também
escalável. Afinal, reproduzir o modelo de negócio em outras regiões e países
também envolve a capacidade da empresa de crescer, preferencialmente de uma
forma saudável. Para tal é necessário que os serviços sejam prestados sem
demandar recursos na mesma proporção que o seu crescimento, usando somente 
uma estrutura básica comum a todos \cite{CassioSpina}.

No livro \textit{A Startup Exuta}, \citeonline{StartupEnxuta} traz outra
definição de \textit{startup} a qual diz que \begin{quote}Startup é uma instituição
humana projetada para criar novos produtos e serviços sob condições de extrema
incerteza\end{quote}. Um contexto no qual toda aprendizagem é válida e deve ser
reutilizada em um ciclo constante de coleta de \textit{feedbacks} dos clientes.

Essa segunda definição, adiciona uma nova característica ao conceito de
\textit{startup} que é o ambiente extremamente incerto no qual essas empresas
estão inseridas. Com base em  \apudonline{Rueda}{GardelinRausp}, esta incerteza
ambiental é perceptível em uma empresa quando há dúvidas, por parte dos gerentes,
quanto a uma série de fatores, entre eles: a viabilidade de futuras tecnologias,
as expectativas de mudanças de consumo e preferências sociais para os produtos
e serviços e possíveis mudanças na legislação.

\subsection{Fases de uma Startup}

De acordo com \citeonline{Blank2013} no seu livro \textit{The four steps to the Epiphany},
o modelo de desenvolvimento que melhor se adapta ao contexto de uma \textit{startup}
é o Modelo de Desenvolvimento do Cliente, apresentado na \autoref{fig:FourStepsToEpiphany}.
\citeonline{Blank2013} defende que este modelo não é um substituto para o Modelo de
desenvolvimento do Produto tradicionalmente adotado nas grandes empresas, mas sim
um complemento que se concentra na compreensão dos problemas e nas necessidades
do cliente. Este modelo é caracterizado por quatro fases, sendo elas:

\begin{figure}[h]
  \centering
  \includegraphics[keepaspectratio=true,scale=0.6]{figuras/epiphany.eps}
  \caption{Modelo de desenvolvimento do Cliente\label{fig:FourStepsToEpiphany} \source{Blank2013}}
\end{figure}

\begin{description}
    \item[Passo 1: Descoberta do cliente] O objetivo desta etapa é descobrir quem
    são os clientes para o produto proposto e se o problema que está sendo solucionado
    é de fato importante para eles. Nesse sentindo, a primeira ideia e requisitos
    sobre o produto originam-se na verdade a partir dos fundadores da \textit{startup},
    e é papel do Time de Desenvolvimento do Cliente verificar se existe clientes
    e um mercado para tal visão.
    \item[Passo 2: Validação do cliente] Esta etapa tem por objetivo montar um
    roteiro de vendas repetível para as equipes de marketing e vendas seguirem adiante.
    O propósito desse roteiro é guiar a equipe com base em experiências de campo nas
    quais a venda foi bem sucedida para os primeiros clientes. Esta validação mostra
    que foi encontrado um conjunto de clientes interessados e um mercado que reage
    positivamente ao produto.
    \item[Passo 3: Criação do cliente] Com base no sucesso das vendas iniciais, esta
    etapa dedicasse a criar uma demanda no usuário final. Nesta etapa o produto já foi
    validado e agora os investimentos em marketing podem ser altos, permitindo que a
    empresa controle sua taxa de queima de caixa e proteja seus ativos.
    \item[Passo 4: Construção da Empresa] É a etapa na qual a empresa faz a transição da
    sua equipe de Desenvolvimento do Cliente, orientada para aprendizado e descoberta,
    para departamentos formais de vendas, marketing, etc. Esses departamentos, por sua
    vez, devem se concentrar em explorar o sucesso inicial da \textit{startup} no mercado.
\end{description}


\section{Desenvolvimento de Software}

Segundo \citeonline{Despa2014} o processo de construção de um software consiste
em vários estágios distintos, os quais são caracterizados por suas entregas em um
período de tempo específico. Podemos classificar esses estágios nas seguintes
atividades:

\begin{description}
    \item[Pesquisa]{é o estágio destinado a formulação dos requisitos, na qual os
    envolvidos trocam informações a respeito do sistema que está sendo proposto,
    definem metas que devem ser alcançadas e buscam características sobre o mercado,
    o comportamento do usuário e soluções técnicas;}
    \item[Planejamento]{é o estágio destinado a definição dos fluxos que a aplicação
    terá que executar, com base nesses fluxos definisse também as tecnologias que
    serão adotadas, como será organizado o banco de dados e a metodologia que será
    aplicada durante o processo de construção;}
    \item[Design]{é o estágio no qual a interface da aplicação é definida com base
    no seu respectivo contexto. Esta etapa permite uma validação prévia com o cliente,
    na qual costumam surgir novos requisitos e mudanças no sistema. Em geral, é possível
    realizar essa etapa em paralelo a programação;}
    \item[Desenvolvimento]{é a etapa na qual o software é de fato construído. Inicia-se
    com a configuração do ambiente de desenvolvimento, o qual deve estar integrado com o
    ambiente de teste. Nesta fase os desenvolvedores devem estar preocupados em não
    inserir erros no código e deixar comentários que facilitem a manutenção depois;}
    \item[Teste]{é o estágio destinado a identificação de erros tanto de implementação
    quanto de design. Nessa etapa deve ser verificado se o software está de acordo com
    os requisitos planejados, se não está sujeito a falhas de segurança e se existe
    algum problema de usabilidade;}
    \item[Implantação]{é o estágio no qual são realizadas todas as configurações e testes
    necessários para disponibilizar a aplicação em um ambiente ativo para o usuário final.
    Nesta fase, o sistema deve ser integrado com aplicativos de terceiros, caso seja
    necessário, e configurado rotinas de backup;}
    \item[Manutenção]{é o estágio subsequente a parte de implantação, e refere-se a etapa
    na qual são identificados novos erros que possam ter passado despercebidos, adicionadas
    novas funcionalidades e realizado o monitoramento do \textit{log} para a identificação
    de erros e o monitoramento do tráfego, com o intuito de prever possíveis problemas de
    de desempenho que a aplicação possa vir a sofrer.}
\end{description}

Os estágios mencionados acima são geralmente aceitos pela comunidade de software
como os pilares do desenvolvimento de software. Estando, em sua maioria, presentes
nas mais variadas metodologias de desenvolvimento, mesmo que com nomenclaturas
diferentes \cite{Despa2014}.

\subsection{Manifesto Ágil}

O Manifesto Ágil é um conjunto de valores que visam auxiliar a descobrir como
fazer as coisas certas dentro do seu contexto de projeto. Uma das características
que destacam o Agile em relação a outras abordagens de desenvolvimento de software
é foco nas pessoas que realizam o trabalho e como elas trabalham juntas. Abaixo
segue os quatros valores declarados por \citeonline{AgileManifesto} no Manifesto Ágil:

\begin{enumerate}
    \item Indivíduos e interações mais que processos e ferramentas;
    \item Software em funcionamento mais que documentação abrangente;
    \item Colaboração com o cliente mais que negociação de contrato;
    \item Responder a mudanças mais que seguir um plano.
\end{enumerate}

\subsection{Os 12 princípios do software ágil}

Junto com os valores expressos no Manifesto Ágil, o desenvolvimento ágil de software
é guiado pelos doze princípios apresentados a seguir:

\begin{enumerate}
    \item Nossa maior prioridade é satisfazer o cliente, através da entrega adiantada
    e contínua de software de valor;
    \item Aceitar mudanças de requisitos, mesmo no fim do desenvolvimento. Processos
    ágeis se adequam a mudanças, para que o cliente possa tirar vantagens competitivas;
    \item Entregar software funcionando com frequência, na escala de semanas até meses,
    com preferência aos períodos mais curtos;
    \item Pessoas relacionadas à negócios e desenvolvedores devem trabalhar em conjunto
    e diariamente, durante todo o curso do projeto;
    \item Construir projetos ao redor de indivíduos motivados. Dando a eles o ambiente e
    suporte necessário, e confiar que farão seu trabalho;
    \item O Método mais eficiente e eficaz de transmitir informações para, e por dentro
    de um time de desenvolvimento, é através de uma conversa cara a cara;
    \item Software funcional é a medida primária de progresso;
    \item Processos ágeis promovem um ambiente sustentável. Os patrocinadores,
    desenvolvedores e usuários, devem ser capazes de manter indefinidamente, passos
    constantes;
    \item Contínua atenção à excelência técnica e bom design, aumenta a agilidade;
    \item Simplicidade: a arte de maximizar a quantidade de trabalho que não precisou ser
    feito;
    \item As melhores arquiteturas, requisitos e designs emergem de times auto-organizáveis;
    \item Em intervalos regulares, o time reflete em como ficar mais efetivo, então, se
    ajustam e otimizam seu comportamento de acordo.
\end{enumerate}

\subsection{Scrum}

O Scrum pertence a família dos métodos ágeis e é uma metodologia de desenvolvimento
de software iterativa e incremental baseada em \textit{timboxes} e voltada para o
gerenciamento do projeto. Esta metodologia parte da premissa de que o desenvolvimento
de software é muito complexo e imprevisível para ser planejado exatamente com
antecedência. Neste caso, deve ser aplicado o controle empírico do processo afim de
garantir transparência, inspeção e adaptação \cite{Mahnic2005}. Para tal, essa
metodologia utiliza de papéis, eventos, artefatos e regras que os unem. A seguir,
serão explicados cada um desses tópicos individualmente de acordo com as informações
dispostas em \textit{The Scrum Guide} escrito por \citeonline{ScrumGuide}.

\subsubsection{Papéis}

\begin{description}
    \item[\textit{Product Owner}] responsável por representar os interesses de
    todos os envolvidos no projeto e em seu sistema resultante. Deve manter o
    \textit{Product backlog} do projeto priorizado conforme as necessidades do cliente;
    \item[Time de desenvolvimento] é uma equipe autogerenciada, auto-organizada e
    multifuncional, responsável por implementar o \textit{Product backlog}. Os
    membros da equipe são coletivamente responsáveis pelo sucesso de cada iteração
    e do projeto como um todo;
    \item[Scrum master] é o responsável por garantir a execução do processo Scrum.
    Deve ensinar as práticas do Scrum aos envolvidos e incentivá-los a seguir essas
    práticas afim de que a metodologia tenha sucesso na organização.
\end{description}

\subsubsection{Artefatos}

Os artefatos do Scrum são produzidos com o objetivo de fornecer transparência e
permitir inspeção e adaptação, visando maximizar a transparência das informações
principais com o intuito de que toda a equipe tenha o mesmo entendimento a respeito
do artefato.

\begin{description}
    \item[\textit{Product Backlog}] é uma lista ordenada com todas as funcionalidades
    que se sabe que o produto deve contemplar. Esta lista nunca está completa e é
    evoluída ao longo da execução do projeto com a finalidade de atender as necessidades
    do cliente;
    \item[\textit{Sprint Backlog}] é conjunto de itens oriundos do \textit{product backlog},
    com o propósito de serem finalizados durante a execução da respectiva \textit{sprint};
    \item[\textit{Incremento}] é a soma de todos os itens do \textit{product backlog}
    entregues durante a \textit{sprint} e o valor dos incrementos de todas as \textit{sprints}
    anteriores.
\end{description}

\subsubsection{Eventos}

O objetivo dos eventos no Scrum é criar uma regularidade e minimizar a necessidade
de reuniões durante o projeto. Cada evento tem um tempo determinado para acontecer,
não podendo ultrapassar esse limite visando evitar desperdício de tempo durante o
processo. Tais eventos são apresentados adiante:

\begin{description}
    \item[Sprint] é o \textit{timebox} principal do Scrum com duração de 1 mês ou
    menos e o objetivo de criar um produto utilizável e potencialmente liberável.
    Durante a \textit{sprint} são incluídos os outros eventos apresentados adiante;
    \item[Sprint Planning] evento no qual o Time de Desenvolvimento, colaborativamente,
    planeja o que será realizado durante a \textit{sprint} que se inicia. São
    levantados aspectos sobre quais atividades do \textit{backlog} podem ser entregues
    ao final da iteração e como será o trabalho necessário para alcançá-las;
    \item[Daily] são reuniões diárias de no máximo 15 minutos. O objetivo destas
    reuniões é verificar o andamento da \textit{sprint} e planejar o que será executado
    nas próximas 24 horas. Dessa forma visa-se acompanhar e otimizar o andamento da
    \textit{sprint};
    \item[Sprint Review] realizado ao final da \textit{sprint} com o intuito de
    inspecionar e adaptar o \textit{product backlog} com base no que foi entregue;
    \item[Sprint Retrospective] realizado antes do próximo \textit{sprint planning}, é
    uma reunião para a equipe se auto inspecionar e planejar melhorias para a próxima
    \textit{sprint}.
\end{description}

A \autoref{fig:ScrumFramework} visa ilustrar o ciclo de vida do processo Scrum.

    \begin{figure}[h]
      \centering
      \includegraphics[keepaspectratio=true,scale=0.4]{figuras/scrumFramework.eps}
      \caption{Scrum Framework\label{fig:ScrumFramework} \source{WhatIsScrum}}
    \end{figure}

\subsection{Extreme Programming}

A metodologia \gls{XP} é um processo ágil bastante popular que enfatiza a satisfação
do cliente incentivando os desenvolvedores a responder com confiança as mudanças
nos requisitos, ainda que no final do ciclo de vida do projeto. Ela divide o processo
convencional de desenvolvimento de software em blocos menores e mais gerenciáveis visando
minimizar posteriores custos de alteração \cite{Wells2000,Despa2014}. Esta metodologia é
baseada nos seguintes cinco princípios:

\begin{description}
    \item[Simplicidade:] fazer somente o que é necessário e nada a mais que isso. Assim,
    o valor entregue será maximizado e o projeto poderá ser mantido a longo prazo por
    custos razoáveis;
    \item[Comunicação:] todos fazem parte da equipe e devem se comunicar diariamente
    sobre todas as áreas, desde requisitos a código, para que todos possam juntos
    construir a melhor solução;
    \item[Feedback:] o software deve ser demonstrado cedo e por muitas vezes, sendo
    importante ouvir atentamente as observações e realizar as alterações necessárias.
    O processo da equipe deve ser adaptado as necessidades do projeto, e não o contrário;
    \item[Respeito:] todos dão e sentem o respeito que merecem como um membro valioso
    da equipe. Toda contribuição é bem vinda e tem igual valor, independente se ela vem
    do cliente, da gerência ou da equipe de desenvolvimento;
    \item[Coragem:] deve-se sempre dizer a verdade sobre o progresso e as estimativas e
    não ter medo de se adaptar as mudanças sempre que elas ocorrerem.
\end{description}

Nos sub tópicos adiante, serão apresentadas as regras que guiam a metodologia \gls{XP}
em cada área do processo de desenvolvimento de acordo com \citeonline{Wells2000}.

\subsubsection{Planejamento}
\begin{description}
    \item[Histórias de usuário devem ser escritas.] Histórias de usuário são frases escritas
    pelo próprio cliente, sem nenhum linguajar técnico, que visam descrever coisas que o
    sistema deve fazer. Muitas vezes são utilizadas no lugar de um grande documento de
    requisitos e servem para criar estimativas quanto ao tempo durante a reunião de
    planejamento;
    \item[O planejamento do lançamento cria a programação de lançamento.] Deve ser realizada
    uma reunião para traçar um plano de lançamento que estabeleça o projeto geral. Neste
    planejamento são tomadas tanto decisões técnicas quanto decisões de negócios afim de
    definir uma agenda com a qual todos os envolvidos podem se comprometer;
    \item[Faça lançamentos pequenos e frequentes.] A equipe de desenvolvimento deve
    liberar versões iterativas do sistema frenquentemente, pelo menos a cada duas semanas.
    Assim no final de cada de iteração você terá testado e demonstrado a funcionalidade
    para o cliente
    \item[O projeto é dividido em iterações.] O cronograma deve ser dividido
    em várias iterações de 1 a 3 semanas com o intuito de realizar constantemente pequenas
    entregas;
    \item[O planejamento da iteração inicia cada iteração.] Cada iteração deve iniciar com
    o seu planejamento, no qual as histórias de usuários são selecionadas para serem
    implementadas de acordo com a prioridade definida pelo cliente.
\end{description}

\subsubsection{Gerenciamento}
\begin{description}
    \item[Dê ao time um espaço de trabalho aberto e dedicado.] A comunicação é um
    fator muito importante dentro do \acrlong{XP}. Aproximar a equipe fisicamente e
    remover barreiras físicas entre eles incentiva-os a trabalhar colaborativamente
    como um conjunto no qual todos tem igual valor e contribuições;
    \item[Defina um ritmo sustentável.] O objetivo do \gls{XP} é conseguir um software
    mais completo, testado, integrado e pronto para produção a cada iteração. Software
    incompleto ou com erros representa uma quantidade desconhecida de esforços futuros
    que não podem ser mensurados. Nessa perspectiva, a metodologia prega que se as
    histórias planejadas para a iteração não puderem ser concluídas, o ideal é fazer
    um novo planejamento da iteração visando definir o que será entregue até o final.
    Mesmo que falte somente um dia para terminar a iteração, o melhor é focar toda a
    equipe em uma única tarefa completa, do que terminar a iteração com várias tarefas
    incompletas;
    \item[Comece todos os dias com uma reunião em pé.] De acordo com \citeonline{Wells2000},
    uma reunião longa para apresentar os resultados é um desperdício de tempo dos
    desenvolvedores. O ideal é fazer reuniões diárias para comunicar problemas, discutir
    soluções e promover o foco da equipe. Essas reuniões devem ser feitas em círculos, de
    forma que todos se sintam a vontade para contribuir, e em pé para evitar que a reunião
    se prolongue por muito tempo;
    \item[Deve-se medir a velocidade do projeto.] A velocidade do projeto deve ser
    medida com o propósito de auxiliar no planejamento das próximas iterações. A proposta
    do \gls{XP} é estimar de forma simples a dificuldade de implementação das histórias
    e planejar a iteração de acordo com o ritmo que a equipe consegue produzir;
    \item[Mova as pessoas.] Se apenas uma pessoa da sua equipe é capaz de trabalhar
    em determinada área e essa pessoa sair ou estiver sobrecarregada, o progresso do seu
    projeto será consideravelmente reduzido. É importante a realização de treinamentos
    cruzados afim de evitar ilhas de conhecimento. Os desenvolvedores devem ser
    incentivados a trabalhar em seção diferente do código pelo menos em uma parte da
    iteração;
    \item[Conserte o \gls{XP} quando ele não funcionar.] Algumas alterações serão
    necessárias na metodologia para atender as particularidades do seu projeto. O
    recomendado é começar seguindo as regras do \gls{XP} e realizar reuniões de
    retrospectiva para conversar com a equipe e discutir o que não está funcionando
    e como o processo pode ser melhorado.
\end{description}

\subsubsection{Design}
\begin{description}
    \item[Simplicidade.] Faça sempre a coisa mais fácil que possa funcionar a seguir.
    Optar por um design simples significa menos tempo de implementação e é sempre mais
    barato substituir um código complexo por um mais simples agora, antes que seja
    perdido muito mais tempo. É subjetivo medir a simplicidade, mas o recomendado pela
    metodologia é que o código seja testável, compreensível, navegável e explicável.
    Testável significa que você pode escrever testes de unidade e aceitação para
    identificar rapidamente algum problema. Navegável é a qualidade de poder encontrar
    facilmente o que você precisa, isso é refletido em bons nomes, no uso correto de
    heranças, etc. A definição de compreensível é óbvia, mas altamente subjetiva. Por
    isso o código também deve ser explicável, ou seja, compreensível de uma forma que
    todos que já trabalham com o código consigam entender, e explicável de forma que
    novos desenvolvedores também possam entender o código;
    \item[Escolha uma metáfora do sistema.] A metáfora do sistema é algo simples que
    possa facilmente explicar como é o design do sistema sem precisar ler várias
    documentações para tal. O intuito dela é orientar o desenvolvimento e facilitar
    a inserção de novos membros;
    \item[Use cartões \gls{CRC} para o design.] O maior do valor dos cartões é fazer
    com que a equipe se afaste do pensamento processual e enfoque mais nos objetos.
    Permitir que toda a equipe trabalhe na construção dos cartões promove um melhor
    um melhor projeto do sistema e um maior entendimento da equipe;
    \item[Crie soluções de pico para reduzir o risco.] Quando existe dificuldade
    de tomar decisões sobre problemas técnicos ou designs difíceis, o ideal é criar
    algo muito simples que explore potenciais soluções, mesmo que este não atenda
    todos os problemas e resolva todas as preocupações. Muito provavelmente, este
    programa de teste não será suficientemente bom para manter e será jogado fora.
    No entanto, o objetivo dessa ação é reduzir o risco de um problema técnico e
        aumentar a confiabilidade sobre a história de usuário;
    \item[Nenhuma funcionalidade deve ser adicionada cedo.] Muitos desenvolvedores
    são tentados adicionar funcionalidades visando que é fácil adicioná-las agora e
    que estas poderão ser utilizadas futuramente, ou que a arquitetura se tornará
    muito mais flexível. No entanto, a realidade é de utilizar muito pouco essas
    funcionalidades extras e deixar a arquitetura mais complexa do que o necessário
    para o problema. Dessa forma, o \gls{XP} recomenda olhar sempre para as
    necessidades atuais e não para as necessidades futuras;
    \item[Refatorar sempre e sempre que possível.] O \gls{XP} prega que você deve
    refatorar o código sempre que possível, remover duplicações, códigos obsoletos,
    alterar códigos que sejam difíceis de manejar, etc. Manter o código limpo e
    conciso economiza tempo ao longo da vida do software.
\end{description}

\subsubsection{Codificação}
\begin{description}
    \item[O cliente está sempre disponível.] O ideal é que o cliente esteja sempre
    no mesmo espaço que o time de desenvolvimento participando de todas as fases
    do projeto. É importante que esse cliente seja um especialista da organização
    afim de ajudar a garantir que as funcionalidades sejam implementadas conforme
    o desejado;
    \item[O código deve ser escrito de acordo com os padrões acordados.] Manter o
    código em um padrão acordado pela equipe facilita o entendimento e a refatoração
    do mesmo. Além de incentivar a propriedade coletiva do código;
    \item[Escreva os testes unitários primeiro.] Criar os testes antes do código
    torna mais rápido e fácil a criação do próprio código. Ajuda o desenvolvedor a
    considerar o que realmente precisa ser feito e a estabelecer os requisitos por
    meio dos testes;
    \item[Todo o código de produção é programado em pares.] A programação em pares
    aumenta a qualidade do código sem afetar o tempo de entrega. A ideia é
    contra-intuitiva, mas duas pessoas que trabalham em um único computador adicionam
    tanta funcionalidade quanto duas que trabalham separadamente, com o diferencial
    da qualidade do código ser melhor;
    \item[Apenas um par integra o código por vez.] A integração paralela do código
    significa que dois ou mais códigos que ainda não foram testados juntos serão
    integrados acreditando-se que está tudo bem. Contudo, essa integração se aplica
    também a suíte de testes e, se não existe um conjunto de testes consolidados,
    problemas de integração aconteceram sem detecção. Por isso a importância de
    integrar um código por vez, afim de gerar sempre uma versão mais estável e clara
    do sistema;
    \item[Integre-se frequentemente.] Os desenvolvedores devem estar alinhados e
    trabalhando sempre na versão mais atualizada do código. Isso evita futuros
    problemas de integração e reduz o tempo de desenvolvimento uma vez que evita
    a alteração em uma linha de código que está obsoleta;
    \item[Configure um computador de integração dedicado.] O propósito é que todos os
    lançamentos do software para produção sejam feitos pelo mesmo computador. Assim,
    todos os envolvidos podem acompanhar o que está sendo integrado e a garantia de
    que o código está sempre atualizado;
    \item[Propriedade coletiva.] Qualquer desenvolvedor deve ser livre pra alterar
    qualquer linha do código, seja para corrigir um \textit{bug}, refatorar ou
    implementar uma nova funcionalidade. Toda a equipe é responsável pelo design da
    aplicação, e não cabe a uma única mente - o arquiteto chefe, por exemplo - definir
    como será feito esse design.
\end{description}

\subsubsection{Teste}
\begin{description}
    \item[Todo o código deve ter testes unitários.] No \gls{XP} um código só pode ser
    lançado se todos os testes unitários estiverem juntos a nova versão. Caso seja
    identificado algum código sem teste, ele deve ser implementado imediatamente. Assim,
    você protege o código contra danos acidentais;
    \item[Todo o código deve passar por todos os testes unitários antes de ser lançado.]
    Dessa forma visa-se garantir que todas as funcionalidades estarão funcionando no
    sistema disponibilizado para produção;
    \item[Quando um \textit{bug} é encontrado, testes são criados.] O teste deve ser
    criado com o intuito de sinalizar a existência do \textit{bug} e fazer os
    programadores se moverem o mais rápido possível para repará-lo;
    \item[Os testes de aceitação são executados constatemente e a pontuação publicada.]
    Os testes de aceitação são testes de caixa preta com o intuito de verificar que
    o sistema está cumprindo as exigências definidas nas histórias de usuários. Dessa
    forma o \gls{XP} exige que o desenvolvimento de software tenha um relacionamento
    muito mais próximo com o controle de qualidade.
\end{description}

A \autoref{fig:XPFramework} visa ilustrar o ciclo de vida da metodologia \acrlong{XP}.

    \begin{figure}[h]
      \centering
      \includegraphics[keepaspectratio=true,scale=0.8]{figuras/xpFramework.eps}
      \caption{Ciclo de vida da metodologia  Extreme Programming\label{fig:XPFramework} \source{Wells2000}}
    \end{figure}

\subsection{Desenvolvimento de software em \textit{startups}}

Do ponto de vista da engenharia, o desenvolvimento de software em \textit{startups}
trabalha em um contexto difícil para os processos de software seguirem uma metodologia
prescritiva. Diante dos diversos fatores associados, o contexto específico de cada
\textit{startup} torna o processo de desenvolvimento único e um grande desafio para
a Engenharia de Software \cite{Giardino2016}. Por isso, são necessárias pesquisas para
investigar e apoiar as atividades da engenharia em \textit{startups}, afim de orientar
os profissionais na tomada de decisões visando minimizar os riscos do fracasso
comercial. Contudo, apesar do tamanho impressionante do ecossistema de
\textit{startups}, a pesquisa sobre Engenharia de Software em \textit{startups}
apresenta uma lacuna\cite{Giardino2016}.

Os processos de desenvolvimento ágil são focados na solução e respondem bem a
"como" criar produtos rapidamente, no entanto, eles não respondem bem a "quais"
produtos construir \cite{Bosch2013}. Nesse sentido, eles se aplicam bem a situações
em que o problema é bem compreendido e a solução não é. Mas para uma \textit{startup}
nem o problema nem a solução são muito bem compreendidos \cite{Bosch2013}.

Dessa forma, as \textit{startups} costumam optar por práticas voltadas ao produto que
lhe permitam ter uma equipe flexível com fluxos de trabalho facilmente
adaptáveis a novas direções de acordo com o público-alvo \cite{Paternoster2014}.
Nesse contexto, a literatura indica práticas de desenvolvimento de software
orientadas ao mercado, nas quais é enfatizado a importância de dedicar tempo
a objetivos estratégicos específicos \cite{Paternoster2014}. Assim, as
\textit{startups} enfrentam diariamente um troca entre engenharia de alta
velocidade e de alta qualidade, não apenas em aspectos como design da arquitetura
mas em aspectos multifacetados como gerenciamento de projetos, testes, etc.
\cite{Giardino2016}

\section{Escalabilidade}

Segundo \citeonline {PlatformEcosystems}, escalabilidade refere-se ao grau em que
o desempenho funcional e financeiro de um subsistema é independente de seu tamanho.
Podendo manter seu desempenho e função retendo todas as propriedades desejadas sem
um aumento correspondente em sua complexidade interna
\apud{EngineeringSystems}{PlatformEcosystems}. Para Tiwana, isso implica em duas
proposições pertinentes sobre a escalabilidade: a primeira é de que escalabilidade não
se refere exclusivamente a aumentar a escala de um sistema, como habitualmente
pensamos, mas também a capacidade deste sistema de se contrair conforme a necessidade.
A segunda proposição diz que o desempenho de um sistema escalável pode significar
tanto a sua capacidade técnica de desempenho quanto a sua capacidade financeira.

Assim, a escalabilidade reflete a capacidade do software em crescer e evoluir
conforme as demandas do usuário. Dessa forma, é importante analisar se o
modelo de negócios da empresa depende do crescimento desse sistema. Uma vez que
não existe essa dependência, a escalabilidade não se torna um requisito básico,
podendo um software não escalável funcionar bem com uma capacidade limitada
\cite{ConceptaScalability}.

No texto \textit{The Importance of Scalability In Software Design}
da \citeonline[tradução nossa]{ConceptaScalability}, a empresa Concepta
\footnote{Saiba mais em: \url{https://conceptainc.com}} diz que \begin{quote}
A escalabilidade é um componente essencial do software corporativo. Priorizá-lo
desde o início leva a menores custos de manutenção, melhor experiência do usuário
e maior agilidade.\end{quote}\footnote{Texto original: \textit{Scalability is an
essential component of enterprise software. Prioritizing it from the start leads
to lower maintenance costs, better user experience, and higher agility.}} Nessa
abordagem apresentada pela Concepta, a escalabilidade vai além de atender a um
grande volume de demandas de requisições, e passa a fazer parte da experiência
do usuário além de influenciar positivamente a manutenção desse software. Seguindo
essa visão de escalabilidade apresentada pela Concepta, existem três fatores que
afetam significantemente a escalabilidade de softwares:

  \begin{enumerate}
    \item A capacidade de usuários simultâneos ou conexões possíveis.
    Aumentar essas capacidade deve ser tão simples quanto disponibilizar mais
    recursos ao software;
    \item A capacidade de armazenamento, principalmente para aplicações
    que apresentam muitos dados não estruturados. A escolha do tipo do banco de
    dados que será usado e o uso de uma indexação adequada são fatores que
    podem influenciar diretamente a escalabilidade;
    \item O próprio código. Desenvolvedores inexperientes tendem a ignorar
    o fato de que o código deve ser escrito para que possa ser facilmente
    adicionado ou modificado, sem a necessidade de refatorar o código antigo.
    Bons desenvolvedores evitam a duplicação de esforços, reduzindo o tamanho
    e a complexidade geral da base do código.
  \end{enumerate}

Priorizar esses pontos no início da construção do software, traz agilidade e
benefícios que garantem o crescimento saudável do sistema.

\subsection{Escalabilidade vertical}

A primeira e mais direta forma de lhe dar com o aumento da demanda de um
sistema é por meio da escalabilidade vertical. Esta refere-se a maximização
dos recursos de uma única unidade visando expandir sua capacidade de lhe dar
com o aumento de carga.  Pensando em hardware, isto significa aumentar o poder
de processamento e memória do servidor no qual o sistema é executado. Em termos
de software esse tipo de escalabilidade está relacionado a otimização dos código
e algoritmos utilizados \cite{FreshGuide2012}.

O escalamento vertical costuma ser bem compreendido e arquiteturalmente direto,
com o auxílio da engenharia ele pode ajudar o sistema a atender seus requisitos
de carga de trabalho \cite{InterSystems2019}. No entanto, ele possui as seguintes
limitações:

\begin{enumerate}
    \item{Mesmo com a utilização de servidores poderosos com mais de cem núcleos
    e terabytes de memória, existe um limite para a quantidade de conexões
    simultâneas que um sistema pode manter;}
    \item{Maior capacidade de hardware implica em custos mais elevados para
    adquiri-los sempre que for necessário expandir a capacidade, resultando em
    um crescimento exponencial sobre o investimento necessário para manter tal
    escalabilidade;}
    \item{O escalonamento vertical exige um planejamento cuidadoso antes de
    realizar o mesmo, o que pode ser feito para contextos relativamente estáticos,
    contudo, para contextos dinâmicos é difícil prever o futuro;}
    \item{Não há elasticidade na escala vertical, uma vez que se aumente a
    capacidade do sistema, não é possível diminuí-la caso a sua demanda permita.
    Isto acaba gerando maiores custos por excesso de capacidade;}
    \item{A escalabilidade vertical exige que o seu software seja adaptado para
    tal. Por exemplo, escalar 128 núcleos é pouca utilidade se a aplicação pode
    lhe dar com no máximo 32 processos.}
\end{enumerate}

\subsection{Escalabilidade Horizontal}

A escalabilidade horizontal refere-se ao incremento de recursos pela adição
de unidades ao sistema. Em comparação a escalabilidade vertical, a escalabilidade
horizontal significa adicionar várias unidades de menor capacidade ao invés de
uma única unidade com altíssima capacidade. Isso implica na distribuição da carga
de trabalho entre as várias unidades existentes \cite{FreshGuide2012}.

Essa abordagem de escalonamento permite um crescimento gradual do sistema, em
contrapartida a substituição abrupta realizada no escalonamento vertical. Dessa
forma ela se torna financeiramente mais vantajosa. Além de se adaptar muito bem
a infraestruturas virtuais e de nuvem, fornecendo facilmente provisionamento
a medida que a carga de trabalho aumenta ou desativando unidades caso a demanda
diminua \cite{InterSystems2019}.

\subsection{Bancos NoSQL}

Atualmente, as aplicações têm trabalhado com um volume de dados cada vez maior.
E nesse contexto, as propriedades \gls{ACID} implementadas pelos bancos de dados relacionais,
tornam-se muito onerosas mediante as necessidades específicas desses aplicativos.
Assim, o custo associado ao dimensionamento dos bancos de dados tradicionais em
frente ao crescente volume de dados se torna muito caro \cite{Gajendran}.

Diante desta situação, desenvolvedores buscam soluções alternativas para o
armazenamento de dados, dentre elas estão os bancos não-relacionais. O termo
\gls{NoSQL} refere-se a um conjunto de banco de dados que não possuem as características
tradicionais de um banco de dados relacional \cite{Gajendran}. Estes bancos são projetados
para atender uma larga escala de armazenamento de dados e processamento paralelo
massivo em cima desses dados \cite{NewEraOfDatabases}.

Para obter essa escalabilidade, os bancos \gls{NoSQL} \textit{"abriram mão"} de
manter a consistências dos dados, como ocorre nos bancos de dados relacionais,
resultando em sistemas conhecidos como \gls{BASE}. Estes sistemas não possuem
transações no sentido clássico e introduzem restrições no modelo de dados
para permitir melhores esquemas de partição \apud{SurveyNosql}{NewEraOfDatabases}.

Assim eles oferecem um armazenamento relativamente barato e altamente escalonável
para pequenos pacotes de registros, como \textit{logs}, leitura de medidores,
\textit{snapshots} e para dados semi-estruturados ou não-estruturados, como arquivos
de e-mail, \textit{xml} e documentos. A estrutura distribuída desses bancos também
os tornam ideias para processamento massivo de dados em lote \cite{NewEraOfDatabases}.

\subsection{Sistemas monolíticos}

A Arquitetura Monolítica é um padrão de desenvolvimento de software no qual um aplicativo
é criado com uma única base de código, um único sistema de compilação, um único binário
executável e vários módulos para recursos técnicos ou de negócios. Seus componentes
trabalham compartilhando mesmo espaço de memória e recursos formando uma unidade de
código coesa \cite{NatalliaSakovich}.

Esse padrão arquitetural permite um desenvolvimento fácil por ser de conhecimento
da maioria dos desenvolvedores de software e apresentar baixa complexidade para a
execução de tarefas como \textit{deploy}, confecção de testes e compartilhamento
do código. Contudo, a longo prazo começam a surgir dificuldades para manter essa
arquitetura, entre elas estão:

  \begin{enumerate}
    \item Dificuldade em entender e alterar o código que ao longo do tempo se torna
    muito extenso e coeso;
    \item Limitação na agilidade de atualização do software, uma vez que para cada
    pequena alteração é necessário reimplantar o código por completo;
    \item Necessita de muito esforço para adotar uma nova tecnologia, sendo
    necessário adaptar todo o código da aplicação para a nova ferramenta;
    \item O sistema perde a confiabilidade pois a medida que cresce um \textit{bug}
    em uma única parte do código pode interromper todo o software já que todos os
    componentes estão conectados.
  \end{enumerate}

\subsection{Microsserviços}

Segundo \citeonline{Newman2015}, microsserviços são pequenos serviços autônomos que
trabalham juntos. O objetivo desses serviços é tornar o código coeso e focado em
resolver as regras de negócio dentro de um limite específico. As vantagens de adotar
essa arquitetura consistem em:

\begin{enumerate}
    \item{Uso de tecnologias heterogêneas;}
    \item{Maior resiliência do sistema;}
    \item{Facilidade em escalar pequenas partes do sistema, ao invés do sistema por
    completo;}
    \item{Facilidade e maior estabilidade no processo de \textit{deploy};}
    \item{Maior produtividade da equipe, uma vez que se passa a adotar equipes
    menores com bases de código menores;}
    \item{Reusabilidade dos serviços;}
    \item{Otimização da substituibilidade: organizações de médio e grande porte costumam
    possuir sistemas legados, com uma base de código enorme funcionando com tecnologias
    antigas tanto de software quanto de hardware. A substituição desses códigos é um
    processo custoso e arriscado. Com serviços pequenos realizar essa substituição
    se torna um processo mais fácil de gerenciar.}
\end{enumerate}

Para o desenvolvimento de microsserviços é necessário pensar sobre o que deve ser
exposto e o que deve ser ocultado. Se houver muito compartilhamento os serviços de
consumo se acoplam às nossas representações internas, diminuindo a nossa autonomia
e consequentemente a nossa capacidade de realizar alterações \cite{Newman2015}.

\subsection{Escalabilidade de software em \textit{startups}}

Segundo \apudonline{Stampfl2013}{Kotsch2017}, o fator mais crucial para o potencial
de crescimento de uma \textit{startup}, tanto operacionalmente quanto financeiramente,
é a escalabilidade. \textit{Startups} que crescem rapidamente e respondem bem as
mudanças provavelmente superarão as grandes empresas. A exemplo, está a \textit{startup}
Airbnb que foi fundada no ano de 2008, com sede em San Francisco - Califórnia, e
rapidamente se expandiu para outros países. Dentre outros fatores envolvidos, a
escalabilidade foi provavelmente um dos ativos mais decisivos no crescimento da
empresa, uma vez que permitiu a ela aumentar sua capacidade em proporções superiores
ao aumento do custo.

A tecnologia é frequentemente considerada um bom investimento porque tem a capacidade
de permitir a escalabilidade. No entanto, se um empresário quer escalar rapidamente
pode cair na armadilha da escalabilidade prematura \cite{Kotsch2017}. Esta escalabilidade
prematura pode estar associada ao produto, ao cliente, à equipe, ao modelo de negócios
ou às finanças \cite{StartupGenome2011}.

O relatório \textit{Startup Genome Report Extra on Premature Scaling}\footnote{O relatório
está disponível \href{https://a4389177-39da-4622-a867-e7d6f48a3333.filesusr.com/ugd/fde30c_6b9eb284e987474899a746989086d8ee.pdf}
{neste link}.}
confeccionado pelo projeto Startup Genome Project apresenta alguns fatores de inconsistência
que sinalizam o escalamento prematuro, dentre eles está o fato de muitas \textit{startups}
investirem na escalabilidade da tecnologia antes mesmo deste produto está ajustado ao
mercado \cite{StartupGenome2011}.

\citeonline{Brikman2015} defende que a equipe de \gls{TI} dentro de uma \textit{startup}
deve estar preocupada com a escalabilidade em dois aspectos. O primeiro refere-se a
escalabilidade do código, ou seja, dimensionar as práticas de codificação para lidar
com mais desenvolvedores, mais código e maior complexidade. O segundo aspecto é dimensionar
o desempenho do código para lidar com mais usuários, mais tráfego e mais dados.

\begin{quotation}{Brikman2015}{translated=true}
    Escalar uma \textit{startup} é um pouco como mudar de marcha em um carro com câmbio manual.
    Escalar muito cedo é como mudar para uma marcha alta enquanto o carro esta se movendo
    em baixa velocidade: as marchas vão ranger e você pode parar completamente. Escalar
    tarde demais é como pisar no pedal do acelerador enquanto se mantém em marcha baixa:
    o motor será bastante pressionado, empurrado para o vermelho, e se continuar por muito
    tempo ele superaquecerá sem atingir a velocidade máxima. Você precisa escalar e mudar
    de marcha no momento certo para manter as coisas funcionando sem problema.\footnote{
    Texto original: \textit{Scaling a startup is a bit like shifting gears in a car with a
    manual transmission. Scaling too early is like shifting into a high gear while the car
    is moving at a low speed: the gears will grind and you may stall out completely. Scaling
    too late is like flooring the gas pedal while staying in low gear: you put a lot of stress
    on the engine, push it into the red, and if you keep it up too long, you'll overheat
    without ever hitting top speed. You have to scale and you have to shift gears at the right
    time to keep things moving smoothly.}}
\end{quotation}

Seguindo a analogia apresentada por Brikman, as \textit{startups} em suas primeiras fases
do ciclo de vida devem se preocupar com a escalabilidade do código, que seria o equivalente
as marchas baixas, ajustá-las antes de se preocupar com as marchas mais altas pois estas,
por sua vez, podem nunca nem ser atingidas.
