\begin{resumo}
    A arquitura monolítica traz dentro da engenharia de software um histórico de sistemas legados e
    equipes frustradas com a complexidade de manutenção desses sistemas. Do outro lado a arquitetura
    de microsserviços é capaz de fornecer uma série de benefícios almejados por diversas empresas.
    Esse contexto leva essas empresas a optarem por adotar ou migrar seus sistemas para a
    arquitetura de microsserviços. Contudo, essa migração realizada em desacordo com os objetivos de
    negócio da empresa e sem o planejamento adequado, leva esses sistemas ao fracasso do modelo
    arquitetural. Tendo isso em vista, o presente estudo realizou uma pesquisa exploratória sobre os
    estilos arquiteturais monolítico e microsserviços, construindo um mapa mental com os fatores que
    afetam cada modelo arquitetural. Ao final, analisou-se esses fatores em quatro casos de estudo de
    empresas que optaram por migrar de um modelo arquitetural para o outro. Por fim, concluiu-se que
    a arquitetura monolítica é indicada para descoberta do domínio mas que essa arquitetura tende a
    perder manutenibilidade e evolucionabilidade a medida que a base de código cresce. Enquanto que
    a arquitetura de microsserviços tende a ser mais sustentável, porém apresenta um alto custo e
    exige domínio sobre a problemática e as tecnologias.
 \vspace{\onelineskip}
    
 \noindent
  \textbf{Palavras-chave}: Arquitetura de Software. Microsserviços. Sistemas monolíticos.
\end{resumo}
