\begin{resumo}
    % Nos últimos anos houve um crescente surgimento de empresas denominadas
    % \textit{startups}. Essas empresas possuem como objetivo buscar, no menor
    % tempo possível, um modelo de negócios repetitível e escalável que gere
    % inovação no mercado. Nesse contexto nasceu a \textit{startup} Rua Dois
    % com o propósito de inovar o mercado imobiliário. Contudo, durante
    % a sua busca por inovação e escalabilidade, o processo de desenvolvimento
    % de software foi conturbado. Levando a algumas decisões técnicas que
    % dificultaram a evolução do software na mesma velocidade que era desejada
    % pela a empresa. Consequentemente, foi necessário simplificar toda a arquitetura
    % do software desenvolvido com o intuito de adaptar-se melhor ao atual contexto
    % da \textit{startup}. O presente trabaho realizará uma análise comparativa
    % entre as duas arquiteturas adotadas na empresa com o intuito de avaliá-las
    % e definir possíveis diretrizes que orientem a Rua Dois em relação as suas
    % expectativas de desenvolvimento de software escalável.
 \vspace{\onelineskip}
    
 \noindent
  \textbf{Palavras-chave}: Desenvolvimento de software. Escalabilidade.
    Estudo de caso. \textit{Startup}.
\end{resumo}
