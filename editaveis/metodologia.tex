\chapter{Metodologia}
\label{sec:Metodologia}

\section{Levantamento bibliográfico}

Para referenciamento deste trabalho foram realizados estudos acerca de
\textit{startups}, desenvolvimento de software e aspectos relacionados
a escalabilidade de um sistema por meio de livros, Google Scholar, dentre
outras fontes. Os tópicos abordados dentro de cada tema foram selecionados
com base na sua respectiva importância dentro do assunto e no seu envolvimento
com o contexto aprensentado. Individualmente, os temas citados são bem explorados e
abordados pela comunidade acadêmica, contudo, quando relacionados os artigos e
estudos a respeito tendem a ser mais escassos.

Publicações e palestras de empresas de referencia voltadas ao desenvolvimento de
software no contexto de \textit{startups} também foram utilizadas com o intuito de
contribuir com perspectivas mais atuais a respeito do tema.

\section{Metodologia de pesquisa}
\label{sec:MetodologiaPesquisa}

A proposta deste trabalho é correlacionar técnicas de desenvolvimento de software escalável com as
fases de vida de uma \textit{startup}. Para tal, foi aplicado o METODO X, visando estabelecer essa
correlação. Esta seção visa descrever os objetos de análise, o método aplicado e
as etapas envolvidas no desenvolvimento de toda a pesquisa. 

\subsection{Objetos de análise}
\label{sec:ObjetosDeAnalise}


Os objetos de análise do presente estudo consistem em:

    \begin{description}
        \item [Ciclo de vida de uma \textit{startup}] Fases pelas quais uma \textit{startup} passa
            desde o seu nascimento até a sua estabilização no mercado.
        \item [Técnicas de escalabilidade] Técnicas e boas práticas aplicadas no mercado em busca da
            construção de sistemas escaláveis e/ou com a capacidade para tal.
    \end{description}

\subsection{Método de correlação}

\subsection{Etapas do desenvolvimento}
\label{sec:EtapasDoDesenvolvimento}

A etapas adotadas na construção da seguinte pesquisa foram:

    \begin{description}
        \item [Etapa 1] Definição do método de análise a ser aplicado;
        \item [Etapa 2] Coleta de dados a respeito dos objetos de análise;
        \item [Etapa 3] Aplicação do método de análise sobre os dados coletados;
        \item [Etapa 4] Análise dos resultados obtidos;
        \item [Etapa 5] Classificação das técnicas de escalabilidade;
        \item [Etapa 6] Validação ??.
    \end{description}

A seguir segue a descrição de cada etapa.

\subsubsection{Etapa 1: Definição do método de análise a ser aplicado}

Nesta etapa, foi definido o método de análise utilizado, o qual tem como objetivo de estabelecer
a correlação entre entre as técnicas de escalabilidade elencadas e as fases do cicla de vida de uma
\textit{startup}.

\subsubsection{Etapa 2: Coleta de dados a respeito dos objetos de análise}
\label{sec:Etapa2}

Mediante o método de análise escolhido, foram coletados por meio de pesquisa bibliográfica as
informações necessárias acerca dos objetos de análise definidos na \autoref{sec:ObjetosDeAnalise}.
Dessa forma obteve-se as características sobre os objetos de análise pertinentes a aplicação do método.

\subsubsection{Etapa 3: Aplicação do método de análise sobre os dados coletados}

Com os dados em mãos, nessa etapa aplicou-se o méotodo de análise proposto visando obter a
correlação entre os objetos analisados.

\subsubsection{Etapa 4: Análise dos resultados obtidos}

A partir dos resultados obtidos na etapa anterior, realizou-se uma análise afim de compreender os
mesmos e analisar a coerência com o que era esperado.

\subsubsection{Etapa 5: Classificação das técnicas de escalabilidade}

Por fim, classificou-se as técnicas de escalabilidade elencadas em relação as fases da
\textit{startup} usando como base os resultados obtidos nas etapas anteriores. Obtendo assim, as
recomendações almejadas no objetivo geral deste trabalho, o qual está disposto na \autoref{sec:ObjetivoGeral}.

\section{Planejamento das atividades}

Esta seção destina-se a descrever o planejamento realizado visando atingir as etapas definidas
na \autoref{sec:EtapasDoDesenvolvimento} e com base na metodologia de pesquisa descrita
na \autoref{sec:MetodologiaPesquisa}.

Mediante o escopo da pesquisa a ser realizada, planejou-se a execução para um total
de 7 semanas, iniciando na metade do mês de Março/2020 e indo até a primeira semana de
Maio/2020. No \autoref{quad:Cronograma} estão dispostas as atividades que pretende-se cumprir
ao longo deste período.

\begin{quadro}
    \caption{Cronograma das atividades\label{quad:Cronograma}}
    \begin{tabular}{ | c | l | }
    \hline
    \textbf{Semana} &
        \textbf{Atividades} \\ \hline
        1 & Definição do método de análise a ser aplicado \\ \hline
        2 & Coleta de dados a respeito dos objetos de análise \\ \hline
        3 & Aplicação do método de análise sobre os dados coletados \\ \hline
        4 & Análise dos resultados obtidos e classificação das técnicas de escalabilidade \\ \hline
        6 & Revisão e escrita do documento \\ \hline
        7 & Apresentação \\ \hline
    \end{tabular}
\end{quadro}

\section{Ferramentas}

As ferramentas que pretende-se usar para a execução deste trabalho são:

\begin{description}
    \item[Draw.io] editor gráfico online que permite a construção de processos e
    desenhos relevantes ao conteúdo apresentado;
    \item[Google Planilhas] ferramenta para construção de tabelas e gráficos.
    A qual pretende-se usar para realizar análises sobre os dados coletados.
    \item[Trello] quadro interativo que auxilia na organização de projetos. O mesmo foi utilizado
        para gerenciar as etapas e tarefas alocadas para cada semana de execução do projeto,
        permitindo acompanhar o andamento do mesmo.
\end{description}

