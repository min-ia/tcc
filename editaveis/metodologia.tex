\chapter{Metodologia}
\label{sec:Metodologia}

Com o propósito de atingir o objetivo descrito na \autoref{sec:ObjetivoGeral}, o presente trabalho
realizou uma pesquisa exploratória acerca do tema abordado juntamente com a análise de alguns casos
práticos de empresas que fizeram a migração de uma arquitetura monolítica para uma arquitetura de
microsserviços, ou vice e versa. Assim, o desenvolvimento deste estudo baseou-se nas seguintes etapas:

\begin{description}
    \item [Etapa 1] Levantamento teórico inicial acerca dos objetos de análise diante dos
        referências teóricos adotados;
    \item [Etapa 2] Fichamento e agrupamento dos pontos de vista descobertos na Etapa 1, afim de
        encontrar as perspectivas comuns e divergentes entre os autores estudados;
    \item [Etapa 3] Síntese acerca dos aspectos levantados sobre cada estilo arquitetural;
    \item [Etapa 4] Descrição dos casos de estudo relevantes para o tema;
    \item [Etapa 5] Análise dos casos de estudo diante das perspectivas teóricas levantadas;
    \item [Etapa 6] Análise final comparando os dois modelos arquiteturais com base nos referências
        teóricos adotados e nos casos práticos estudados.
\end{description}

A seguir serão apresentados os objetos de análise, as ferramentas utilizadas e a descrição de cada
etapa adotada.

\section{Objetos de análise}
\label{sec:ObjetosDeAnalise}

Os objetos de análise pertinentes para o presente estudo consistem em:

    \begin{description}
        \item [Arquitetura monolítica:] estilo arquitetural tradicionalmente adotado em diversas
            empresas, no qual se detém uma base única de código.
        \item [Arquitetura de microsserviços:] estilo arquitetural distribuído, largamente adotado
            nos últimos anos por diversas empresas.
    \end{description}

\section{Ferramentas}

As ferramentas utilizadas para auxiliar na confecção do presente estudo foram:

\begin{description}
    \item[Miro] editor gráfico online que permite a construção de diagramas, mapa mentais, entre
        outros recursos gráficos relevantes ao conteúdo apresentado;
    \item[Google Planilhas] ferramenta para construção de tabelas.
    \item[Trello] quadro interativo que auxilia na organização de projetos utilizando do modelo
        Kanban.
\end{description}

\section{Descrição das etapas}
\subsection{Etapa 1 - Levantamento teórico}

A fundamentação teórica deste trabalho visou explorar os temas de arquitetura de software,
estilo arquitetural monolítico e estilo arquitetural de microsserviços por meio de livros, Google Scholar e
outras fontes de informação. Os tópicos abordados foram selecionados com base na sua respectiva
relevância dentro do assunto estudado e tiveram embasamento principalmente nos autores Martin Fowler,
Sam Newman, Mark Richards e Neal Ford, dentre outros.

\subsection{Etapa 2 - Perspectivas}
\label{met:perspectivas}

Mediante a fundamentação teórica construída na Etapa 1, levantou-se os principais pontos acerca dos
estilos arquiteturais abordados que eram tratados pelos autores estudados, para tal utilizou-se da
ferramenta Trello com o intuito de organizar os referências teóricos e destacar o ponto de vista de
cada autor. A partir daí, utilizou-se \textit{post-its} para agrupar pontos de vista em comum entre
os autores e classificar quais eram os aspectos positivos e negativos de cada arquitetura.

Diante dos agrupamentos formados, levantou-se a definição de \textbf{perspectivas} para o presente
estudo, as quais referem-se a formas diferentes de olhar para a arquitetura. Assim, elencou-se as
seguintes perspectivas:

\begin{description}
    \item[Problemática a ser resolvida:] fatores que influenciam o sistema sob a perspectiva do
        problema e do contexto no qual a aplicação está sendo construída;
    \item[Recursos necessários:] fatores que influenciam o sistema sob a perspectiva de quais
        recursos: financeiro, humano e tempo, são necessários para o desenvolvimento de uma
        aplicação nos modelos arquiteturais analisados;
    \item[Características arquiteturais:] fatores que influenciam o sistema sob a perspectiva de
        capacidade da arquitetura de prover algum aspecto que possa ser relevante para a empresa,
        como escalabilidade;
    \item[Manutenibilidade:] fatores que influenciam o sistema sob a perspectiva cotidiana de
        manutenção de cada modelo arquitetural, como processos rotineiros de \textit{deploy} e
        integração contínua e comunicação das equipes;
    \item[Evolucionabilidade:] fatores que influenciam o sistema sob a perspectiva de tomada de
        decisão dentro de cada modelo arquitetural.
\end{description}

Nota-se que a fundamentação teórica foi apresentada sob a organização dessas perspectivas com o
intuito de facilitar a compreensão do presente estudo.

\subsection{Etapa 3 - Síntese}
